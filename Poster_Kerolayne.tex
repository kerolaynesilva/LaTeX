%%Modelo criado por R\'egis (c) 2011
%% http://latexbr.blogspot.com
%% twitter: @rg3915
\documentclass{sciposter}% especifica o tipo de documento (neste caso é um poster)
\usepackage[brazil]{babel} %define o idioma utilizado
%\usepackage[latin1]{inputenc}% este pacote se aplica somente ao windows
\usepackage[utf8]{inputenc} % para trabalhar com linux (sistema operacional)
\usepackage[centertags]{amsmath} %amplia os comandos matematicos
\usepackage{hyperref,amsfonts,multirow,multicol}% palavras separadas por virgulas significas varios pacotes
\hypersetup{pdfpagelayout=SinglePage} %abre a pagina em modo simples
\usepackage{graphicx,colortbl}%permite inserir figuras
\usepackage{lipsum} %sugest\~ao de texto
\usepackage{geometry}
\geometry{paperwidth=90cm,paperheight=120cm,centering,
    textwidth=77cm,textheight=115cm,left=3cm,top=3cm}% corresponde as dimensoes do poster
\pagestyle{plain}  
\frenchspacing
%\usepackage{pstricks,pst-solides3d}
%\usepackage{tikz}

%%Transformando tudo em branco
\renewcommand{\thesection}{\textcolor{white}{\arabic{section}}} % estabelece a cor e fonte das seções
\renewcommand{\thesubsection}{\textcolor{white}{\arabic{section}.\arabic{subsection}}}
\addto\captionsbrazil{\renewcommand{\bibname}{\textcolor{white}{Refer\^encias}}}

%%Definindo cores
\definecolor{BoxCol}{RGB}{32,178,170} %cor para ver mais cores visite o site  Colour Lovers

%%Define o caminho das figuras, v\~a�lido somente para o comando \includegraphics
\graphicspath{{figuras/}}
%********************************************************************
%%Teoremas
\usepackage{theorem}
\theorembodyfont{\normalfont\upshape}
\newtheorem{ex}{Exemplo}
\newenvironment{defn}[1][Defini\c{c}\~ao]{\textbf{#1:}\
}%
%********************************************************************
%%Definindo de novos comandos

\newcommand{\tituloA}[1]{\emph{\textbf{\color{white}{#1}}}}
\newcommand{\tituloB}[1]{\emph{\textbf{\color{blue}{#1}}}}
\newcommand{\R}{\mathbb{R}}


%********************************************************************
\begin{document}

%%titulo
\colorbox{BoxCol}{
  \begin{minipage}{\textwidth}
    \color{white}{
      \begin{center}
        \huge{\textbf{\vspace{1cm} \\
          EQUAÇÕES DIFERENCIAIS ORDINÁRIAS APLICADAS NO ESTUDO DE CIRCUITOS ELÉTRICOS  %insira o titulo aqui
        \vspace{1cm} \\}}
      \end{center}
    }
  \end{minipage}
}

\qquad %espa\c{c}o

%Nome, instituto, email, logo
\title{}
\author{\textbf{Kerolayne Meneses da Silva e Taciana Oliveira Souza†}}
%\footnote{kerolaynemenesesphb@hotmail.com}}}
\institute{FEELT, FAMAT}
\email{\texttt{\, \,  kerolaynemenesesphb@hotmail.com}}
\leftlogo[0.9]{ufulogo} %logotipo da universidade
\rightlogo[1.4]{pet} %logotipo da universidade

\qquad %espa\c{c}o

\maketitle %gera titulo

\colorbox{BoxCol}{
  \begin{minipage}{\textwidth}
    \begin{center}
      \vspace{0.3 cm}
      %digite o nome do congresso aqui
      \Large{ \tituloA{ VII Semana de Iniciação Científica e Tecnológica da UFU e Mostra de Trabalhos do PET}}
      \vspace{0.3cm} 
    \end{center}
  \end{minipage}
}

\quad

%Texto em 2 colunas
\begin{multicols}{2}{

%Paragrafo.
\setlength{\parindent}{2em}

\section*{\tituloA{INTRODU\c{C}\~AO}}
%\PARstart{O} 
Da necessidade de descrever sistemas ou fenômenos naturais surgiu a  modelagem matemática, isto é, a descrição matemática de um sistema ou fenômeno feita, em muitas situações, por meio de equações envolvendo as variáveis atuantes. A partir da simulação de um fenômeno, são obtidos dados experimentais com os quais se elabora um modelo matemático correspondente. Muitas vezes este modelo inclui a taxa de variação de uma variável atuante, resultando nas chamadas Equações Diferenciais Ordinárias (EDO's).


As EDO's são subdivididas segundo sua ordem, onde se destacam as equações de primeira e segunda ordem.  Destacamos a vasta gama de aplicações das EDO's dentro da modelagem de sistemas, fato que pode ser verificado nas referências \cite{BOYCE}, \cite{PACHECO}, \cite{SOUZA} e \cite{ZILL}.  Neste trabalho apresentamos aplicações das EDO's lineares de primeira e segunda ordens no estudo de circuitos elétricos.

%Tais grupos de EDO's podem ser ramificados ainda em vários tipos de equações, destacando: equações separáveis, equações lineares, equações exatas e equações homogêneas e equações lineares de segunda ordem.
\section*{\tituloA{METODOLOGIA}}

O método utilizado no desenvolvimento deste trabalho consistiu no estudo de livros e artigos relacionados ao tema proposto. Encontros semanais foram realizados nos quais a estudante  exp\^os sobre os t\'opicos estudados,  promovendo discuss\~oes com a orientadora para sanar as d\'uvidas surgidas. Para a apresenta\c{c}\~ao semanal de resultados, a estudante entrou ainda em contato com o ambiente TeXnicCenter (linguagem LateX) que permite a elabora\c{c}\~ao de textos t\'ecnico-cient\'ificos.


%\subsection*{\tituloB{subse\c{c}\~ao}}

\section*{\tituloA{RESULTADOS}}

Um circuito elétrico é formado por fontes de energia, fios condutores e alguns elementos elétricos como resistores, capacitores e indutores. O circuito é considerado fechado quando a corrente elétrica que sai do terminal positivo da fonte de energia, percorre todos os componentes e volta no polo oposto da fonte. Para compreender o funcionamento de um circuito, é necessário saber conceitos básicos de eletricidade que podem ser encontrados em \cite{CHARLES}.

\subsection*{CIRCUITOS RL}

O circuito RL é constituído por um resistor, um indutor e é alimentado por uma fonte. Por ele percorre uma corrente $I(t)$, como pode ser verificado na Figura 1 abaixo.

\begin{figure}[!htb]\label{circuito}
\centering
\includegraphics[scale=0.6]{CircuitoA.pdf}
\begin{center} 
Figura 1: Circuito RL
\end{center}
\end{figure}

Aplicando a Lei das Malhas de Kirchhoff e considerando uma corrente circulante no sentido horário, encontra-se a seguinte expressão:

\begin{equation}
V(t) = L \frac{dI}{dt} + RI.
\end{equation}

\begin{ex}:
Suponha que, no circuito da Figura 1, $V = 30 V$, $L = 0.1H$ e $R = 50\Omega$. Determine a corrente $I(t)$ sabendo que $I(0) = 0$.

\noindent {\bf Solução:}
Como a equação (1) é uma EDO linear de primeira ordem, tem-se o seguinte fator integrante $\displaystyle u(t) = e^{\int \frac{R}{L}dt} = e^{\frac{Rt}{L}}$, multiplicando esse fator pela equação (1) resulta em

\begin{center}
$\displaystyle V(t)e^{\frac{Rt}{L}} = e^{\frac{Rt}{L}} \frac{dI}{dt} + e^{\frac{Rt}{L}} \frac{RI}{L}$.
\end{center}
Resolvendo a equação linear acima, obtém-se

%\begin{center}
%$\displaystyle I(t) = V + Ke^{\frac{-Rt}{L}}$
%\end{center}
\begin{center}
$\displaystyle I(t) = 30 + Ke^{-500t}$.
\end{center}
Como $I(0) = 0$, $K = -30$. Portanto, a solução é

\begin{center}
$\displaystyle I(t) = 30 - 30e^{-500t}$.
\end{center}
\end{ex}

\subsection*{CIRCUITOS RLC}

O circuito RLC é semelhante ao RL, difere apenas pela adição de um capacitor, como pode observado na Figura 2.

\begin{figure}[!htb]\label{circuito}
\centering
\includegraphics[scale=0.6]{CircuitoB.pdf}
\begin{center}
Figura 2: Circuito RLC
\end{center}
\end{figure}

Aplicando a Lei das Malhas de Kirchhoff e considerando uma corrente circulante no sentido horário, encontra-se a seguinte expressão:

\begin{equation}
V(t) = L \frac{d^2Q}{dt^2} + R\frac{dQ}{dt} + \frac{Q}{C}.
\end{equation}

\begin{ex}:
Suponha que, no circuito da Figura 2, $V = 300 V$, $L = \displaystyle \frac{5}{3}H$, $R = 10\Omega$ e $C = \displaystyle \frac{1}{30}f$. Determine a corrente $I(t)$ e a carga $Q(t)$ sabendo que $Q(0) = 0$ e $Q'(0) = 0$.

\noindent {\bf Solução:} Pela equação (2), tem-se:

\begin{equation}
300 = \frac{5}{3}\frac{d^2Q}{dt^2} + 10\frac{dQ}{dt} + 30Q.
\end{equation}
Aplicando a transformada de Laplace na equação (3), adotando $\mathcal{L}\{Q(t)\} = q$:

\begin{center}
$\displaystyle \frac{5}{3}(s^2q - sq(0) - q'(0)) + 10(sq - q(0)) + 30q = \frac{300}{s}$.
\end{center}
Isolando $q$ para encontrar as transformadas elementares:

\begin{center}
$\displaystyle q = \frac{180}{s(s^2 + 6s + 18)}$.
\end{center}
Pelo método das frações parciais,

\begin{center}
$\displaystyle q = \frac{10}{s} - \frac{10(s + 3)}{(s + 3)^2 + 9} - \frac{30}{(s + 3)^2 + 9}$.
\end{center}

A equação acima apresenta as seguintes transformadas inversas elementares: $\displaystyle \mathcal{L}^{-1}\ \Big\{\frac{a}{s} \Big\} = a$, $\displaystyle \mathcal{L}^{-1}\ \Big\{\frac{a}{s^2 + a^2} \Big\} = sen(at)$ e $\displaystyle \mathcal{L}^{-1}\ \Big\{\frac{s}{s^2 + a^2} \Big\} = cos(at)$. Aplicando a transformada inversa, obtém-se:

\begin{equation}
Q(t) = \mathcal{L}^{-1}\{q(s)\} = 10 - 10e^{-3t}cos(3t) -10e^{-3t}sen(3t).
\end{equation} 

Pela expressão $I(t) = \displaystyle \frac{dQ}{dt}$, tem-se

\begin{equation}
I(t) = 60e^{-3t}sen(3t).
\end{equation} 
\end{ex}


\section*{\tituloA{CONCLUS\~AO}}

%Destaca-se  a introdu\c{c}\~ao dos estudantes \`a linguagem \LaTeX  (usada na digita\c{c}\~ao de textos matem\'aticos), a qual foi utilizada na elabora\c{c}\~ao deste p\^oster.
%Conclui-se que a inicia\c{c}\~ao cient\'ifica proposta desempenhou um papel importante no desenvolvimento e amplia\c{c}\~ao da forma\c{c}\~ao  da estudante.

Durante este estudo, pode-se perceber a importância das equações diferenciais na modelagem de situações reais, como na análise de circuitos elétricos. Com isso, foi necessário o contato uma nova linguagem de programação, o LaTeX, que tornou possível a escrita de termos matemáticos com maior facilidade. Portanto, esse projeto proporcionou uma nova perspectiva de análise e resolução de sistemas, proporcionando ferramentas essenciais para a resolução de problemas do cotidiano humano.

\section*{\tituloA{REFER\^ENCIAS}}
\begingroup
\renewcommand{\section}[2]{}
\begin{thebibliography}{99}

\bibitem{BOYCE} BOYCE,  W. E;  Di PRIMA, R.C.,  {\it Equações Diferenciais Elementares e Problemas de Valores de Contorno.}  8ed. Guanabara: LTC Editora, 2006.

\bibitem{CHARLES} ALEXANDER, C. K.; SADIKU, M. N. O., {\it Fundamentos  de  Circuitos  Elétricos.} 5ed. São Paulo: McGraw-Hill Ltda 2008.

\bibitem{PACHECO} PACHECO, A. L. S.,  {\it Transformadas de Laplace: algumas aplicações.}  Monografia, 2011.

\bibitem{SOUZA} RODRIGUES, B. P.; SOUZA, T. O., {\it Equações Diferenciais Ordinárias: do Beisebol à Eletricidade.} Horizonte Científico, v.10, n. 1, 2016.

%\bibitem{SODRE} SODRÉ, U. {\it Equações Diferenciais Ordinárias.}  Notas de aula,  2003.

%\bibitem{SPIEGEL} SPIEGEL, M. R.  {\it Transformadas de Laplace.}   McGraw-Hill do Brasil, 1971.

%\bibitem{STEWART} STEWART, J. {\it Cálculo, Volume 2.} 7ed. São Paulo: Cengage Learning, 2013.

%\bibitem{ZILL} ZILL, D. G., CULLEN, M. R. {\it Equações Diferenciais.} Makron Books, 2001.
\bibitem{ZILL} ZILL, D. G., {\it Equações Diferenciais com Aplicações em Modelagem.} São Paulo: Cengage Learning, 2011.


\end{thebibliography}
\endgroup
}
\end{multicols}

\end{document} 