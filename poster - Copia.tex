%\title{LaTeX Portrait Poster Template}
%%%%%%%%%%%%%%%%%%%%%%%%%%%%%%%%%%%%%%%%%
% a0poster Portrait Poster
% LaTeX Template
% Version 1.0 (22/06/13)
%
% The a0poster class was created by:
% Gerlinde Kettl and Matthias Weiser (tex@kettl.de)
% 
% This template has been downloaded from:
% http://www.LaTeXTemplates.com
%
% License:
% CC BY-NC-SA 3.0 (http://creativecommons.org/licenses/by-nc-sa/3.0/)
%
%%%%%%%%%%%%%%%%%%%%%%%%%%%%%%%%%%%%%%%%%

%----------------------------------------------------------------------------------------
%	PACKAGES AND OTHER DOCUMENT CONFIGURATIONS
%----------------------------------------------------------------------------------------

\documentclass[a0,portrait]{a0poster}


\usepackage{multicol} % This is so we can have multiple columns of text side-by-side
\columnsep=100pt % This is the amount of white space between the columns in the poster
\columnseprule=3pt % This is the thickness of the black line between the columns in the poster

\usepackage[svgnames]{xcolor} % Specify colors by their 'svgnames', for a full list of all colors available see here: http://www.latextemplates.com/svgnames-colors

\usepackage{times} % Use the times font
%\usepackage{palatino} % Uncomment to use the Palatino font
\usepackage{tcolorbox}
\usepackage{graphicx} % Required for including images
\graphicspath{{figures/}} % Location of the graphics files
\usepackage{booktabs} % Top and bottom rules for table
\usepackage[font=small,labelfont=bf]{caption} % Required for specifying captions to tables and figures
\usepackage{amsfonts, amsmath, amsthm, amssymb} % For math fonts, symbols and environments
\usepackage{wrapfig} % Allows wrapping text around tables and figures
\usepackage[portuges, brazil, english]{babel}  %define o idioma utilizado
%\usepackage[latin1]{inputenc}% este pacote se aplica somente ao windows
\usepackage{graphicx,colortbl}%permite inserir figuras
\usepackage{lipsum} %sugest\~ao de texto

\usepackage{geometry}
\geometry{paperwidth=90cm,paperheight=120cm,centering, textwidth=77cm, top=4cm}% corresponde as dimensoes do poster
\pagestyle{plain}  
\frenchspacing
%\usepackage{pstricks,pst-solides3d}
%\usepackage{tikz}
\definecolor{BoxCol}{RGB}{198,229,217} %c0or para ver mais cores visite o site Colour Lovers

\begin{document}

%----------------------------------------------------------------------------------------
%	POSTER HEADER 
%----------------------------------------------------------------------------------------

% The header is divided into two boxes:
% The first is 80% wide and houses the title, subtitle, names, university/organization and contact information
% The second is 20% wide and houses a logo for your university/organization or a photo of you
% The widths of these boxes can be easily edited to accommodate your content as you see fit


%
\noindent\makebox[\textwidth][l]{%
  \hspace{-\dimexpr\oddsidemargin+1in}%
  \colorbox{BoxCol}{%
    \parbox{\dimexpr\paperwidth-2\fboxsep}{\begin{center}
		\VeryHuge \color{Black} \textbf{EQUA\c{C}\~OES DIFERENCIAIS ORDIN\'ARIAS APLICADAS NO ESTUDO DE CIRCUITOS EL\'ETRICOS} \color{Black}\\[1.5cm] % Title
\Huge\textit{VII Semana de Inicia\c{c}\~ao Cient\'ifica e Tecnol\'ogica da UFU e Mostra de Trabalhos do PET}\\[2.0cm] % Subtitle
\huge \textbf{Kerolayne Meneses da Silva e  Taciana Oliveira Souza$^{\dagger}$}\\[0.5cm] % Author(s)
\huge Universidade Federal de Uberl\^andia, FEELT, FAMAT$^{\dagger}$\\[0.4cm] % University/organization
\Large \texttt{kerolaynesilvaphb@gmail.com} 
\end{center}}%		
  }%
}

\vspace{1cm} % A bit of extra whitespace between the header and poster content

%----------------------------------------------------------------------------------------

\begin{multicols}{2} % This is how many columns your poster will be broken into, a portrait poster is generally split into 2 columns


%----------------------------------------------------------------------------------------
%	INTRODUCTION
%----------------------------------------------------------------------------------------

\color{Black} % SaddleBrown color for the introduction


\section*{INTRODU\c{C}\~AO}

Da necessidade de descrever sistemas ou fen\^omenos naturais surgiu a  modelagem matem\'atica, isto \'e, a descri\c{c}\~ao matem\'atica de um sistema ou fen\^omeno feita, em muitas situa\c{c}\~oes, por meio de equa\c{c}\~oes envolvendo as vari\'aveis atuantes. A partir da simula\c{c}\~ao de um fen\^omeno, s\~ao obtidos dados experimentais com os quais se elabora um modelo matem\'atico correspondente. Muitas vezes este modelo inclui a taxa de varia\c{c}\~ao de uma vari\'avel atuante, resultando nas chamadas Equa\c{c}\~oes Diferenciais Ordin\'arias (EDO's).
As EDO's s\~ao subdivididas segundo sua ordem, onde se destacam as equa\c{c}\~oes de primeira e segunda ordem.  Destacamos a vasta gama de aplica\c{c}\~oes das EDO's dentro da modelagem de sistemas, fato que pode ser verificado nas refer\^encias \cite{BOYCE}, \cite{PACHECO}, \cite{SOUZA} e \cite{ZILL}.  Neste trabalho apresentamos aplica��es das EDO's lineares de primeira e segunda ordens no estudo de circuitos el\'etricos.

%Tais grupos de EDO's podem ser ramificados ainda em v�rios tipos de equa��es, destacando: equa��es separ�veis, equa��es lineares, equa��es exatas e equa��es homog�neas e equa��es lineares de segunda ordem.

\color{Black} % DarkSlateGray color for the rest of the content

\setlength{\parindent}{2em}

\section*{METODOLOGIA}

O m\'etodo utilizado no desenvolvimento deste trabalho consistiu no estudo de livros e artigos relacionados ao tema proposto. Encontros semanais foram realizados nos quais a estudante  exp\^os sobre os t\'opicos estudados,  promovendo discuss\~oes com a orientadora para sanar as d\'uvidas surgidas. Para a apresenta\c{c}\~ao semanal de resultados, a estudante entrou ainda em contato com o ambiente TeXnicCenter (linguagem LateX) que permite a elabora\c{c}\~ao de textos t\'ecnico-cient\'ificos.


%\subsection*{\tituloB{subse\c{c}\~ao}}

\section*{RESULTADOS}

\setlength{\parindent}{2em}

Um circuito el\'etrico \'e formado por fontes de energia, fios condutores e alguns elementos el\'etricos como resistores, capacitores e indutores. O circuito \'e considerado fechado quando a corrente el\'etrica que sai do terminal positivo da fonte de energia, percorre todos os componentes e volta no polo oposto da fonte. Para compreender o funcionamento de um circuito, \'e necess\'ario saber conceitos b\'asicos de eletricidade que podem ser encontrados em \cite{CHARLES}.

\subsection*{CIRCUITOS RL}

O circuito RL \'e constitu\'ido por um resistor, um indutor e \'e alimentado por uma fonte. Por ele percorre uma corrente $I(t)$, como pode ser verificado na Figura 1 abaixo.

\begin{minipage}[b]{0.90\linewidth}

\centering
\includegraphics[scale=0.6]{CircuitoA.pdf}\ 
\begin{center}
Figura 1: Circuito RL
\end{center} 
\end{minipage}

Aplicando a Lei das Malhas de Kirchhoff e considerando uma corrente circulante no sentido hor\'ario, encontra-se a seguinte express�o:

\begin{equation}
V(t) = L \frac{dI}{dt} + RI.
\end{equation}

\noindent {\bf Exemplo:}
Suponha que, no circuito da Figura 1, $V = 30 V$, $L = 0.1H$ e $R = 50\Omega$. Determine a corrente $I(t)$ sabendo que $I(0) = 0$.

\noindent {\bf Solu\c{c}\~ao:}
Como a equa��o (1) \'e uma EDO linear de primeira ordem, tem-se o seguinte fator integrante $\displaystyle u(t) = e^{\int \frac{R}{L}dt} = e^{\frac{Rt}{L}}$, multiplicando esse fator pela equa��o (1) resulta em

\begin{center}
$\displaystyle V(t)e^{\frac{Rt}{L}} = e^{\frac{Rt}{L}} \frac{dI}{dt} + e^{\frac{Rt}{L}} \frac{RI}{L}$.
\end{center}
Resolvendo a equa��o linear acima, obt\'em-se

%\begin{center}
%$\displaystyle I(t) = V + Ke^{\frac{-Rt}{L}}$
%\end{center}
\begin{center}
$\displaystyle I(t) = 30 + Ke^{-500t}$.
\end{center}
Como $I(0) = 0$, $K = -30$. Portanto, a solu\'c{c}\~ao \'e

\begin{center}
$\displaystyle I(t) = 30 - 30e^{-500t}$.
\end{center}

\subsection*{CIRCUITOS RLC}

O circuito RLC \'e semelhante ao RL, difere apenas pela adi\c{c}\~ao de um capacitor, como pode observado na Figura 2.

\begin{minipage}[b]{0.90\linewidth}

\centering
\includegraphics[scale=0.6]{CircuitoB.pdf}\ 
\begin{center}
Figura 1: Circuito RL
\end{center}
 
\end{minipage}

Aplicando a Lei das Malhas de Kirchhoff e considerando uma corrente circulante no sentido hor\'ario, encontra-se a seguinte express\~ao:

\begin{equation}
V(t) = L \frac{d^2Q}{dt^2} + R\frac{dQ}{dt} + \frac{Q}{C}.
\end{equation}

\noindent {\bf Exemplo:}
Suponha que, no circuito da Figura 2, $V = 300 V$, $L = \displaystyle \frac{5}{3}H$, $R = 10\Omega$ e $C = \displaystyle \frac{1}{30}f$. Determine a corrente $I(t)$ e a carga $Q(t)$ sabendo que $Q(0) = 0$ e $Q'(0) = 0$.

\noindent {\bf Solu\c{c}\~ao:} Pela equa\c{c}\~ao (2), tem-se:

\begin{equation}
300 = \frac{5}{3}\frac{d^2Q}{dt^2} + 10\frac{dQ}{dt} + 30Q.
\end{equation}
Aplicando a transformada de Laplace na equa\c{c}\~ao (3), adotando $\mathcal{L}\{Q(t)\} = q$:

\begin{center}
$\displaystyle \frac{5}{3}(s^2q - sq(0) - q'(0)) + 10(sq - q(0)) + 30q = \frac{300}{s}$.
\end{center}
Isolando $q$ para encontrar as transformadas elementares:

\begin{center}
$\displaystyle q = \frac{180}{s(s^2 + 6s + 18)}$.
\end{center}
Pelo m\'etodo das fra\c{c}\~oes parciais,

\begin{center}
$\displaystyle q = \frac{10}{s} - \frac{10(s + 3)}{(s + 3)^2 + 9} - \frac{30}{(s + 3)^2 + 9}$.
\end{center}

A equa\c{c}\~ao acima apresenta as seguintes transformadas inversas elementares: $\displaystyle \mathcal{L}^{-1}\ \Big\{\frac{a}{s} \Big\} = a$, $\displaystyle \mathcal{L}^{-1}\ \Big\{\frac{a}{s^2 + a^2} \Big\} = sen(at)$ e $\displaystyle \mathcal{L}^{-1}\ \Big\{\frac{s}{s^2 + a^2} \Big\} = cos(at)$. Aplicando a transformada inversa, obt\'em-se:

\begin{equation}
Q(t) = \mathcal{L}^{-1}\{q(s)\} = 10 - 10e^{-3t}cos(3t) -10e^{-3t}sen(3t).
\end{equation} 

Pela express\~ao $I(t) = \displaystyle \frac{dQ}{dt}$, tem-se

\begin{equation}
I(t) = 60e^{-3t}sen(3t).
\end{equation} 


%----------------------------------------------------------------------------------------
%	CONCLUSIONS
%----------------------------------------------------------------------------------------

\color{Black} % SaddleBrown color for the conclusions to make them stand out

\section*{CONCLUS\~AO}

%Destaca-se  a introdu\c{c}\~ao dos estudantes \`a linguagem \LaTeX  (usada na digita\c{c}\~ao de textos matem\'aticos), a qual foi utilizada na elabora\c{c}\~ao deste p\^oster.
%Conclui-se que a inicia\c{c}\~ao cient\'ifica proposta desempenhou um papel importante no desenvolvimento e amplia\c{c}\~ao da forma\c{c}\~ao  da estudante.

Durante este estudo, pode-se perceber a import\^ancia das equa\c{c}\~oes diferenciais na modelagem de situa\c{c}\~oes reais, como na an\'alise de circuitos el\'etricos. Com isso, foi necess\'ario o contato uma nova linguagem de programa\c{c}\~ao, o LaTeX, que tornou poss\'ivel a escrita de termos matem\'aticos com maior facilidade. Portanto, esse projeto proporcionou uma nova perspectiva de an\'alise e resolu\c{c}\~ao de sistemas, proporcionando ferramentas essenciais para a resolu\c{c}\~ao de problemas do cotidiano humano.


\section*{REFER\^ENCIAS}
\begingroup
\renewcommand{\section}[2]{}
\begin{thebibliography}{99}

\bibitem{BOYCE} BOYCE,  W. E;  Di PRIMA, R.C.,  {\it Equa\c{c}\~oes Diferenciais Elementares e Problemas de Valores de Contorno.}  8ed. Guanabara: LTC Editora, 2006.

\bibitem{CHARLES} ALEXANDER, C. K.; SADIKU, M. N. O., {\it Fundamentos  de  Circuitos  El\'etricos.} 5ed. S�o Paulo: McGraw-Hill Ltda 2008.

\bibitem{PACHECO} PACHECO, A. L. S.,  {\it Transformadas de Laplace: algumas aplica\c{c}\~oes.}  Monografia, 2011.

\bibitem{SOUZA} RODRIGUES, B. P.; SOUZA, T. O., {\it Equa\c{c}\~oes Diferenciais Ordin\'arias: do Beisebol \`a Eletricidade.} Horizonte Cient\'ifico, v.10, n. 1, 2016.

%\bibitem{SODRE} SODR�, U. {\it Equa��es Diferenciais Ordin�rias.}  Notas de aula,  2003.

%\bibitem{SPIEGEL} SPIEGEL, M. R.  {\it Transformadas de Laplace.}   McGraw-Hill do Brasil, 1971.

%\bibitem{STEWART} STEWART, J. {\it C�lculo, Volume 2.} 7ed. S�o Paulo: Cengage Learning, 2013.

%\bibitem{ZILL} ZILL, D. G., CULLEN, M. R. {\it Equa��es Diferenciais.} Makron Books, 2001.
\bibitem{ZILL} ZILL, D. G., {\it Equa\c{c}\~oes Diferenciais com Aplica\c{c}\~oes em Modelagem.} S�o Paulo: Cengage Learning, 2011.


\end{thebibliography}
\endgroup

%----------------------------------------------------------------------------------------

\end{multicols}

\vspace{5cm}

\noindent\makebox[\textwidth][l]{%
  \hspace{-\dimexpr\oddsidemargin+1in}%
  \colorbox{BoxCol}{%
    \parbox{\dimexpr\paperwidth-2\fboxsep}{\begin{center}
\includegraphics[width=10cm]{logo_proexc}\hspace{5cm}     
\includegraphics[width=10cm]{pet}\hspace{5cm}
\includegraphics[width=7cm]{Imagem4}\\[5cm]
\end{center}}%		
  }%
}

\end{document}

